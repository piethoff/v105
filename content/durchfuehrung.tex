\section{Durchführung}
\label{sec:Durchführung}
Der allgemeine Versuchsaufbau besteht aus einer Billardkugel mit einem Permanentmagneten in ihrem Zentrum.
Diese befindet sich zwischen einem Helmholtzspulenpaar auf einem Luftkissen, sodass Reibungseffekte vernachlässigt werden können.
Zusätzlich ist für den dritten Versuchsteil ein Stroboskop an der Apperatur befestigt, welches auf die Kugel gerichtet ist.
Es können der Strom $I$ durch die Helmholtzspulen und die Frequenz $\nu$ des Stroboskops an der Apperatur abgelesen werden.
Der Abstand der Spulen beträgt \mbox{$d=2x=\SI{0.138}{\meter}$}, der Radius der Spulen ist $R=\SI{0.109}{\meter}$,
die Windungszahl jeder Spule liegt bei $N=\num{195}$, die Kugel hat den Radius $r=\SI{26.78}{\milli\meter}$ und die Masse $M=\SI{142.2}{\gram}$.
%
\subsection{Messung mithilfe der Gravitation}
In diesen Versuchsaufbau wird durch eine aufgesteckte Aluminiumstange mit verstellbarem Gewicht $m$ eine asymmetrische Massenverteilung erzeugt,
sodass durch die Schwerkraft ein Drehmoment auf die Kugel wirkt.
Dieses gilt es nun, durch regeln der Stromstärke $I$ und somit der Stärke des $\symbf{B}$-Feldes, auszugleichen.
Im Gleichgewichtszustand
erhält man über den Abstand $r$ der kleinen Masse vom Zentrum der Kugel, ihr Gewicht und dem vorliegenden $\symbf{B}$-Feld
nach Gleichung \eqref{eq:grav_dip} eine Möglichkeit auf das Dipolmoment in der Kugel zu schließen.
Wichtig ist hierbei, dass die Apperatur mithilfe des an der Grundplatte angebrachten Fischauges zuvor ins Lot gebracht wird.
Der Versuch wird für verschiedene Abstände $r$ der kleinen Masse $m$ zum Zentrum wiederholt.
%
\subsection{Messung mithilfe der Schwingungsdauer}
Die Kugel wird bei eingeschaltetem $\symbf{B}$-Feld um einen kleinen Winkel aus der Ruhelage ausgelenkt.
Darauf hin schwingt die Kugel wie in Gleichung \eqref{eq:schwing} beschrieben.
Für eine bessere Genauigkeit werden 10 Perioden $T$ der Schwingung gestoppt und gemittelt.
Damit auf das Dipolmoment geschlossen werden kann, muss zuerst das Trägheitsmoment $\symbf{J}$ der Kugel und dafür ihre Gesamtmasse bestimmt werden.
Der Versuch wird für verschieden starke $\symbf{B}$-Felder wiederholt.
%
\subsection{Messung mithilfe der Präzession}
Die Kugel wird bei zunächst abgeschaltetem $\symbf{B}$-Feld auf dem Luftkissen in Rotation versetzt, sodass ihre Rotationsachse nicht parallel zur Symmetrieachse des $\symbf{B}$-Feldes liegt.
Die Frequanz des Stroboskops wird auf die Rotationsfrequenz der Kugel angepasst, sodass eine Markierung auf der Kugel statisch erscheint.
Wird das $\symbf{B}$-Feld angeschaltet wirkt ein Drehmoment auf die sich drehende Kugel, welche daraufhin eine Präzessionsbewegung vollführt.
Die Periodendauer $T_P$ dieser Präzession wird gemessen.
Um das Dipolmoment berechnen zu können muss auch hier das Trägheitsmoment $\symbf{J}$ der Kugel zuvor ermittelt werden,
um den Drehimpuls $\symbf{L}$ bestimmen zu können.
Der Versuch wird für verschieden starke $\symbf{B}$-Felder wiederholt.
