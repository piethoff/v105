\section{Zielsetzung}
Das magnetische Moment eines Permanentmagneten soll über drei verschiedene Messmethoden bestimmt werden.
Es soll zudem noch bestimmt werden welche dieser Methoden sich am besten eignet.
\section{Theorie}
\label{sec:Theorie}
Die einfachste Form des Magnetismus ist ein makroskopischer Dipol mit geschlossenen Feldlinien.
Dieser kann von einem Permanentmagneten oder einer Leiterschleife erzeugt werden.
Für eine Leiterschleife errechnet sich das magnetische Moment mit:
\begin{equation}
  \symbf{\mu} = I \symbf{A}
\end{equation}
Das magnetische Moment für einen Permanentmagneten lässt sich nicht ohne weiteres berechnen.
Befindet sich der Dipol in einem äußeren Magnetfeld so wirkt ein Drehmoment,
\begin{equation}
  \symbf{D} = \symbf{\mu}\times \symbf{B},
\end{equation}
auf diesen.
Der Dipol erfährt so lange ein Drehmoment, bis $\symbf{\mu}$ und $\symbf{B}$ parallel sind.
Ein homogenes Magnetfeld wird in Versuchen häufig durch ein Helmholtzspulenpaar erzeugt, da diese Annordnung ein frei 
zugängliches, nahezu homogenes Magnetfeld erlaubt.
Das Magnetfeld des Helmholtzspulenpaares lässt sich unter zu Hilfenahme des Biot-Savart Gesetzes berechnen:
\begin{equation}
\symup{d}\symbf{B}= \frac{\mu_0 I}{4 \pi} \frac{\symup{d\symbf{s}}\times \symbf{r}}{r^3}
\end{equation}
Damit folgt für einen Stromdurchflossenen Leiter mit einer Windung:
\begin{equation}
  \symbf{B}=\frac{\mu_0 I}{2} \frac{R^2}{(R^2 + x^2)^3/2 } \cdot \symbf{e}_x
\end{equation}
Wobei $x$ hier den Abstand zur Symmetrieachse darstellt.
Das Feld in der Mitte dieser Anordnung berechnet sich, dann durch die Überlagerung der beiden von den Spulen ausgehenden 
Feldern.
Folglich:
\begin{equation}
  B(0)= B_1(x)+B_2(-x) = \frac{\mu_0 I R^2}{(R^2+x^2)^3/2}
\end{equation}
%
\subsection{Messung unter Ausnutzung der Gravitation}
Hier wird ein verschiebares Gewicht, mit Masse $m$, auf eine Aluminiumstange gesteckt.
Die Gravitationskraft übt ein Drehmoment
\begin{equation}
  \symbf{D}_g = m (\symbf{r} \times \symbf{g})
\end{equation}
auf die Billiardkugel aus.
Das externe Magnetfeld des Helmholtzspulenpaares übt ebenfalls ein Drehmoment
\begin{equation}
  \symbf{D}_B = \symbf{\mu}_\text{Dipol} \times \symbf{B}
\end{equation}
auf die Kugel aus.
Die Kugel befindet sich in einem Gleichgewichtszustand sofern
\begin{equation}
    m (\symbf{r} \times \symbf{g}) = \symbf{\mu}_\text{Dipol} \times \symbf{B}
\end{equation}
 gilt.
 Da $\symbf{B}$ und $\symbf{g}$ parallel sind, fällt die Winkelabhängigkeit weg und $\mu_{Dipol}$ lässt sich mit
 \begin{equation}
   mgr = B\mu_\text{Dipol}
 \end{equation}
 berechnen.
%
\subsection{Messung mithilfe der Schwingungsdauer}
Für einen magnetischen Dipol innerhalb eines homogenen Magnetfeldes ergibt sich folgende Bewegungsgleichung in 
$\theta$-Richtung:
\begin{align}
    -\lvert\symbf{D}_B\rvert &= J \cdot \frac{\symup{d}^2\theta}{\symup{dt}^2} \notag \\
    -B\mu_\text{Dipol}\cdot \sin{\theta} &= J \cdot \frac{\symup{d}^2\theta}{\symup{dt}^2}
\end{align}
Für kleine Winkel kann die Differentialgleichung linearisiert werden und es ergibt sich eine Lösung mit
\begin{equation}
    T^2 = \frac{4\pi^2J}{\mu_\text{Dipol}}\frac{1}{B}
\end{equation}
als Periodendauer der Schwingung.
%
\subsection{Messung mithilfe der Präzession}
Für ein um die Symmetrieachse rotierendes, magnetisches Dipolmoment ergibt sich folgende Differentialgleichung mit 
\mbox{$\symbf{L}=\symbf{J\omega}$} als Drehmoment:
\begin{equation}
    \symbf{\mu}_\text{Dipol} \times \symbf{B} = \frac{\symup{d}\symbf{L}}{\symup{dt}}
\end{equation}
Diese beschreibt die Präzession des Dipols um die Symmetrieachse des Magnetfeldes.
Eine Lösung besitzt die Präzessionsfrequenz
\begin{equation}
    \Omega_p = \frac{B\mu_\text{Dipol}}{L}
\end{equation}
und somit
\begin{equation}
    \frac{1}{T_p} = \frac{\mu_\text{Dipol}}{2\pi L}B .
\end{equation}
