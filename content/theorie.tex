\section{Zielsetzung}
Das magnetische Moment eines Permanentmagneten soll über drei verschiedene Messmethoden bestimmt werden.
Es soll zudem noch bestimmt werden welche dieser Methoden sich am besten eignet.
\section{Theorie}
\label{sec:Theorie}
Die einfachste Form des Magnetismus ist ein makroskopischer Dipol mit geschlossenen Feldlinien.
Dieser kann von einem Permanentmagneten oder einer Leiterschleife erzeugt werden.
Für eine Leiterschleife errechnet sich das magnetische Moment mit:
\begin{equation}
  \vec{\mu} = I \vec{A}
\end{equation}
Das magnetische Moment für einen Permanentmagneten lässt sich nicht ohne weiteres berechnen.
Befindet sich der Dipol in einem außerem Magnetfeld so wirkt ein Drehmoment,
\begin{equation}
  \vec{D} = \vec{\mu}\times \vec{B} ,
\end{equation}
auf diesen.
Der Dipol erhält so lange eine Drehung bis $\vec{\mu}$ und $\vec{B}$ ausgerichtet sind.
Ein homogenes Magnetfeld wird in Versuchen häufig durch ein Helmholtzspulenpaar erzeugt, da diese Annordnung ein frei zugängliches homogenes Magnetfeld erlaubt.
Das Magnetfeld des Helmholtzspulenpaares lässt sich unter zu Hilfenahme des Biot-Savart Gesetzes berechnen:
\begin{equation}
d \vec{B}= \frac{\mu_0 I}{4 \pi} \frac{d\vec{s}\times \vec{r}}{r^3} .
\end{equation}
Damit folgt für einen Stromdurchflossenen Leiter mit einer Windung :
\begin{equation}
  \vec{B}=\frac{\mu_0 I}{2} \frac{R^2}{(R^2 + x^2)^3/2 } \cdot x .
\end{equation}
Wobei $x$ hier den Abstand zur Symmetrieachse darstellt.
Das Feld in der Mitte dieser Annordnung berechnet sich, dann durch die Überlagerung der Beiden, von den Spulen ausgehenden Felder.
Folglich:
\begin{equation}
  B(0)= B_1(x)+B_2(-x) = \frac{\mu_0 I R^2}{(R^2+x^2)^3/2}
\end{equation}
\subsection{Messung unter Ausnutzung der Gravitation}
Hier wird ein verschiebares Gewicht, mit Masse $m$, auf eine Aluminiumstange gesteckt.
Die Gravitationskraft übt ein Drehmoment
\begin{equation}
  \vec{D_g} = m \cdot(\vec{r} \times \vec{g})
\end{equation}
auf die Billiardkugel aus.
Das externe Magnetfeld des Helmholtzspulenpaares übt ebenfalls ein Drehmoment
\begin{equation}
  \vec{D_B} = \vec{\mu_{Dipol}} \times \vec{B}
\end{equation}
auf die Kugel aus.
Die Kugel befindet sich in einem Gleichgewichtszustand sofern
\begin{equation}
   m \cdot(\vec{r} \times \vec{g}) = \vec{\mu_{Dipol}} \times \vec{B}
\end{equation}
 gilt.
 Da $\vec{B}$ und $\vec{g}$ parallel sind fällt die Winkelabhängigkeit weg und $\mu_{Dipol}$ lässt sich mit
 \begin{equation}
    m \cdot(\vec{r} \cdot \vec{g}) = \vec{\mu_{Dipol}} \cdot \vec{B}
 \end{equation}
 berechnen.
