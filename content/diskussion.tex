\section{Diskussion}
\label{sec:Diskussion}
\begin{table}
  \caption{Ergebnisse der Messung.}
  \centering
  \sisetup{table-format=1.3(4)}
  \begin{tabular}{cSSS}
    \toprule
    {Ergebnisse für Messung mit: } & {Gravitation} & {Schwingung} & {Präzession}\\
    \midrule
    {$\mu_\text{Dipol}$} & 0.440\pm0.071 & 0.640\pm0.304 & 0.434\pm0.029 \\
    \bottomrule
  \end{tabular}
\end{table}
\noindent Aus den hier erhaltenen Werten lässt sich folgern, dass die Messung unter Ausnutzung der Gravitation und die Messung über die Präzession am geeignetsten sind.
Ihre Ungenauigkeiten befinden sich in der gleichen Größenordnung und die Werte liegen nah beieinander.
Bei der Messung über die Präzession sind jedoch viele systematische Fehlerquellen enthalten.
Deshalb ist die Aussagekraft des so bestimmten Werts fragwürdig.
Die Abweichung des durch Schwingungsdauer gemessenen Werts lässt sich einerseits, durch die mögliche Varianz der Auslenkungen erklären, andererseits kann es auch sein, dass die Kugel nicht vollständig in nur einer Ebene geschwungen ist.
Somit folgt, dass die Messung unter Ausnutzung der Gravitation am geeignetsten ist, da sie die geringsten systematischen Fehler aufweist.
